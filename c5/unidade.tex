%!TEX program = xelatex
\documentclass{beamer}
\input{../common.tex}
\usepackage[brazil]{babel} % necessário para o plugin LTeX

\tikzset{cell/.style = {draw, rectangle, minimum width=1.5cm, minimum height=0.5cm}}

\title{MC-202\\
Curso de C --- Parte 5}

\begin{document}

\begin{frame}[plain]
  \titlepage
\end{frame}

\begin{frame}{Problema}
  Como calcular o centroide de um conjunto de pontos?\pause

  \begin{columns}
    \begin{column}{0.35\linewidth}
      \centering
      \begin{tikzpicture}[scale=3.5]
        \tikzset{point/.style={draw, fill=TXTC, circle, minimum size=5pt,inner sep=0pt, TXTC, outer sep=1pt}}
        \foreach \x/\y in {
            0.43820859423888425/0.5602517003765746,
            0.8483245987749247/0.22016547793050834,
            0.2844262312589583/0.009124179818488232,
            0.5467579002856252/0.2413291938846639,
            0.7994009526880929/0.03727842490367028,
            0.710205138633686/0.8567908906108719,
            0.21785056797410773/0.09629433864870829,
            0.16833761034539085/0.38298050190699173,
            0.18241629251581493/0.7897566001417992,
            0.27968382873138964/0.6501116852426414,
            0.6707744695205751/0.3096769327353118,
            0.03652356452291916/0.43932036714464984,
            0.5106877800413567/0.5842482649107987,
            0.9970920384350768/0.6497641576890002,
            0.2702320759435839/0.5964479454120887,
            0.8243564062730954/0.41911196654055516,
            0.4071181705605321/0.7077772949076827,
            0.1605649335744881/0.6992169482393208,
            0.5118710238692077/0.07351064298226107,
            0.3207346183907208/0.96876667648163}{
            \node[point] at (\x,\y) {};
          }
        \pause
        \node[point, fill=AC, AC, minimum size=7pt] at (0.459278, 0.464596) {};
        \pause
      \end{tikzpicture}
    \end{column}
    \begin{column}{0.7\linewidth}
      \uncover<4->{\lstinputlisting{codigos/centroide3.c}}
    \end{column}
  \end{columns}
  \uncover<12->{E se tivermos mais do que \alert{\cod{MAX}} pontos?}
\end{frame}

\begin{frame}[<+->]{Ponteiros}

  \action{Toda informação usada pelo programa está em algum lugar}
  \begin{itemize}
    \item Toda variável tem um \alert{endereço de memória}
      \begin{itemize}
        \item cada posição de um vetor também
        \item cada membro de um registro também
      \end{itemize}
  \end{itemize}

  \bigskip
  \action{Um ponteiro é uma variável que armazena um \alert{endereço}}
  \begin{itemize}
    \item para um tipo específico de informação
      \begin{itemize}
        \item \coda{int}, \coda{char}, \coda{double}, structs declaradas, etc.
      \end{itemize}
  \end{itemize}

  \bigskip
  \action{Exemplos:}
  \begin{itemize}
    \item \coda{int *p;} declara um ponteiro para \alert{\cod{int}}
      \begin{itemize}
        \item seu nome é \coda{p}
        \item seu tipo é \coda{int *}
        \item armazena um endereço de um \coda{int}
      \end{itemize}
    \item \coda{double *q;} declara um ponteiro para \coda{double}
    \item \coda{char *c;} declara um ponteiro para \coda{char}
    \item \coda{struct data *d;} declara um ponteiro para \coda{struct data}
  \end{itemize}
\end{frame}

\begin{frame}[<+->]{Operações com ponteiros}
  \action{Operações básicas:}
  \begin{itemize}
    \item \alert{\cod{\&}} retorna o endereço de memória de uma variável (ex: \coda{\&x})
      \begin{itemize}
        \item ou posição de um vetor (ex: \coda{\&v[i]})
        \item ou campo de uma \cod{struct} (ex: \coda{\&data.mes})
        \item podemos salvar o endereço em um ponteiro (ex: \coda{p = \&x;})
      \end{itemize}
    \item \alert{\cod{*}} acessa o conteúdo no endereço indicado pelo ponteiro
      \begin{itemize}
        \item \coda{*p} onde \coda{p} é um ponteiro
        \item podemos ler (ex: \coda{x = *p;}) ou escrever (ex: \coda{*p = 10;})
      \end{itemize}
  \end{itemize}

  \bigskip

  \begin{columns}
    \begin{column}{0.57\linewidth}
      \lstinputlisting{codigos/exemplo1.c}
    \end{column}
    \begin{column}{0.43\linewidth}
      \centering
      \begin{tikzpicture}
        \node[cell, label=above:{\small \cod{endereco}}] at (0,0) {};
        \node[cell, label=above:{\small \cod{variavel}}, label=-60:{\tiny $127$}] at (2.5,0) {$90$};
        \uncover<7->{\node[cell, label=above:{\small \cod{endereco}}] at (0,0) {$127$};
        \draw[-latex, bend right] (0.3,-0.2) to [bend right=50] (2.5,-0.25); }
      \end{tikzpicture}
    \end{column}
  \end{columns}
\end{frame}

\begin{frame}[<+->]{O que é um vetor em C?}
  \action{Em C, se fizermos \coda{int v[100];}}
  \begin{itemize}
    \item temos uma variável chamada \coda{v}
    \item que é, de fato, do tipo \coda{int * const}
      \begin{itemize}
        \item \coda{const} significa que não podemos fazer \coda{v = \&x;}
        \item i.e., não podemos mudar o endereço armazenado em \coda{v}
      \end{itemize}
    \item e que aponta para o primeiro \coda{int} do vetor
      \begin{itemize}
        \item ou seja, \coda{v == \&v[0]}
      \end{itemize}
    \item de uma região da memória de 100 \coda{int}
      \begin{itemize}
        \item normalmente 400 bytes
      \end{itemize}
    \item dizemos que \coda{v} foi alocado estaticamente
      \begin{itemize}
        \item o compilador fez o trabalho
      \end{itemize}
  \end{itemize}

  \bigskip
  \action{Podemos alocar vetores dinamicamente}
  \begin{itemize}
    \item nós alocamos e nós liberamos a região de memória
    \item do tamanho que desejarmos
  \end{itemize}
\end{frame}

\begin{frame}[<+->]{\cod{sizeof}, \cod{malloc} e \cod{free}}
  \action{\coda{sizeof} devolve o tamanho em bytes de um tipo dado}
  \begin{itemize}
    \item \coda{sizeof(int)} (normalmente) devolve \coda{4}
    \item \coda{sizeof(struct data)} - tamanho da \coda{struct data}
      \begin{itemize}
        \item é a soma dos tamanhos dos seus membros
        \item e possivelmente mais alguns bytes para alinhamento
      \end{itemize}
  \end{itemize}

  \bigskip
  \action{\coda{malloc} aloca dinamicamente a quantidade de bytes informada}
  \begin{itemize}
    \item devolve o endereço inicial da região de memória
      \begin{itemize}
        \item a região é sempre contígua
      \end{itemize}
    \item \coda{malloc(sizeof(struct data))} aloca a quantidade de bytes necessária para representar uma \coda{struct data}
    \item \coda{malloc(10 * sizeof(int))} aloca a quantidade de bytes necessária para representar 10 \coda{int}s
  \end{itemize}

  \bigskip
  \action{\coda{free} libera uma região de memória alocada dinamicamente}
  \begin{itemize}
    \item precisa ser um endereço que foi devolvido por \coda{malloc}
    \item evita que vazemos memória (\emph{memory leak})
  \end{itemize}
\end{frame}

\begin{frame}[<+->]{Aritmética de ponteiros}
  \action{Podemos realizar operações aritméticas em ponteiros:}
  \begin{itemize}
    \item somar ou subtrair um número inteiro
    \item também incremento (\coda{++}) e decremento (\coda{--})
    \item o compilador considera o tamanho do tipo apontado
    \item ex: somar \alert{$1$} em um ponteiro para \alert{\cod{int}} faz com que o endereço pule \alert{\cod{sizeof(int)}} bytes
  \end{itemize}

  \medskip

  \action{\lstinputlisting{codigos/exemplo4.c}}

  \medskip

  \action{\centering
    \begin{tikzpicture}
      \foreach \i in {0,1,...,4} {
      \pgfmathtruncatemacro\result{1000 + 4*\i}
      \node[cell, fill=blue!10, label=-60:{\tiny \cod{\result}}] at (0+1.5*\i,-0.5) {\scriptsize \cod{vetor[\i]}};
      }

      \node[cell, label=below:{\scriptsize \cod{vetor}}] (vetor) at (0,-1.75) {\scriptsize\cod{1000}};
      \draw[-latex] (vetor) to (0,-0.75);

      \node[cell, label=below:{\scriptsize \cod{ponteiro}}] (ponteiro) at (4.5, -1.75) {\scriptsize\cod{1012}};
      \draw[-latex] (ponteiro) to (4.5,-0.75);
    \end{tikzpicture}
  }
\end{frame}

\begin{frame}[<+->]{Ponteiros e vetores}
  \action{Se tivermos um ponteiro \coda{p}, podemos escrever \coda{p[i]}}
  \begin{itemize}
    \item como se fosse um vetor
    \item é o mesmo que escrever \coda{*(p + i)}
  \end{itemize}

  \action{\lstinputlisting{codigos/exemplo3.c}}
\end{frame}

\begin{frame}[<+->]{Organização da memória}
  \action{A memória de um programa é dividida em duas partes:}
  \begin{itemize}
    \item \alert{Pilha:} onde são armazenadas as variáveis
      \begin{itemize}
        \item Em geral, espaço limitado (ex: 8MB)
      \end{itemize}
    \item \alert{Heap:} onde são armazenados os outros dados
      \begin{itemize}
        \item Do tamanho da memória RAM disponível
      \end{itemize}
  \end{itemize}

  \medskip

  \action{Alocação estática (variáveis):}
  \begin{itemize}
    \item O compilador reserva um espaço na pilha
    \item A variável é acessada por um nome bem definido
    \item O espaço é liberado quando a função termina
  \end{itemize}

  \medskip

  \action{Alocação dinâmica:}
  \begin{itemize}
    \item \alert{\cod{malloc}} reserva um número de bytes no heap
    \item Devemos guardar o endereço da variável com um ponteiro
    \item O espaço deve ser liberado usando \alert{\cod{free}}
  \end{itemize}
\end{frame}

\begin{frame}[<+->]{Receita para alocação dinâmica de vetores}
  \begin{itemize}
    \item Incluir a biblioteca \alert{\cod{stdlib.h}}
    \item Declare o ponteiro com o tipo apropriado
      \begin{itemize}
        \item ex: \coda{int *v;}
      \end{itemize}
    \item Aloque a região de memória com \coda{malloc}
      \begin{itemize}
        \item O tamanho de um tipo pode ser obtido com \alert{\cod{sizeof}}
        \item ex: \coda{v = malloc(n * sizeof(int));}
      \end{itemize}
    \item Verifique se acabou a memória comparando com \alert{\cod{NULL}}
      \begin{itemize}
        \item use a função \coda{exit} para sair do programa
        \item ex: \lstinputlisting{codigos/memoria.c}
      \end{itemize}
    \item Libere a memória após a utilização com \alert{\cod{free}}
      \begin{itemize}
        \item ex: \coda{free(v);}
      \end{itemize}
  \end{itemize}
\end{frame}

\begin{frame}{Voltando ao centroide}\pause
  \lstinputlisting[basicstyle=\fontsize{0.29cm}{0.31cm}\selectfont\ttfamily]{codigos/centroide4.c}
\end{frame}

\begin{frame}[<+->]{Ponteiros, vetores e funções}
  \action{Funções}
  \begin{itemize}
    \item não podem devolver vetores
      \begin{itemize}
        \item não podemos escrever \coda{int [] funcao(...)}
      \end{itemize}
    \item mas podem devolver ponteiros
      \begin{itemize}
        \item podemos escrever \coda{int * funcao(...)}
      \end{itemize}
  \end{itemize}

  \bigskip
  \bigskip

  \action{\alert{Nunca} devolva o endereço de uma variável local}
  \begin{itemize}
    \item Ela deixará de existir quando a função terminar
    \item Ou seja, nunca devolva um vetor alocado estaticamente
  \end{itemize}

\end{frame}

\begin{frame}[<+->]{Exercício - Alocando vetor}
  \action{Escreva uma função que dado um \coda{int} \alert{$n$}, aloca um vetor de \coda{double} com \alert{$n$} posições zerado.}

  \bigskip
  \bigskip

  \action{\lstinputlisting[firstline=1, lastline=7]{codigos/vetor.c}}

\end{frame}

\begin{frame}[<+->]{Exercício - Imprimindo vetores}
  \action{Queremos fazer uma função que imprime um vetor}
  \begin{itemize}
    \item para vetores alocados estaticamente ou dinamicamente
  \end{itemize}

  \bigskip
  \action{Como vetores são ponteiros, basta receber um ponteiro!}
  \bigskip

  \action{\lstinputlisting[firstline=9, lastline=14]{codigos/vetor.c}}

  \bigskip

  \begin{columns}
    \begin{column}[t]{0.5\linewidth}
      Alocado dinamicamente
      \action{\lstinputlisting[firstline=16, lastline=18]{codigos/vetor.c}}
    \end{column}
    \begin{column}[t]{0.5\linewidth}
      Alocado estaticamente
      \action{\lstinputlisting[firstline=20, lastline=22]{codigos/vetor.c}}
    \end{column}
  \end{columns}
\end{frame}

\begin{frame}[<+->]{Ponteiros e Structs}
  \action{Frequentemente alocamos uma \coda{struct} dinamicamente}
  \begin{itemize}
    \item Elas serão o elemento básico de muitas das EDs
    \item Teremos o ponteiro para uma \coda{struct}
    \item e precisaremos acessar um dos seus campos...
  \end{itemize}

  \bigskip
  \action{Imagine que temos um ponteiro \coda{d} do tipo \coda{struct data *}}
  \begin{itemize}
    \item acessamos o campo \coda{mes} fazendo \coda{(*d).mes}
      \begin{itemize}
        \item veja o endereço armazenado em \coda{d}
        \item vá para essa posição de memória (onde está o registro)
        \item acesse o campo \coda{mes} deste registro
      \end{itemize}
    \item porém isso é tão comum que temos um atalho: \coda{d->mes}
      \begin{itemize}
        \item significa exatamente o mesmo que \coda{(*d).mes}
        \item é um açúcar sintático do C
      \end{itemize}
  \end{itemize}
\end{frame}

\begin{frame}[<+->]{Exercício}
  \begin{itemize}
    \item Declare uma \coda{struct} que armazena informações de notas de uma
      turma. Essa estrutura deve armazenar o número de alunos, as notas das
      provas e a maior nota.

    \item Depois faça um programa que leia todos os dados e imprima a maior nota.
  \end{itemize}
\end{frame}

\end{document}
